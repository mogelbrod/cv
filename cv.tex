% Swedish CV for Victor Hallberg

\def\name{Victor Hallberg}
\date{2012-02-10}
\title{Curriculum Vitae}
\author{\name}

% {{{ Document setup
\documentclass[a4paper,10pt]{article}

% Text configuration
\usepackage[utf8]{inputenc}
\usepackage[swedish]{babel}
\usepackage{lmodern} % font
%\renewcommand{\familydefault}{\sfdefault} % default family

% Paragraphs
\parindent 0mm
\parskip 0mm

% Margins
\usepackage[top=1.8cm, bottom=1.8cm, left=2.5cm, right=2cm]{geometry}
\newlength{\titleleft}
\newlength{\titlefwidth}
\setlength{\titlefwidth}{\textwidth}
\setlength{\titleleft}{-0.5cm}
\addtolength{\titlefwidth}{0.5cm}

% Header and text formatting
\usepackage{titlesec}

\titleformat{\section}{\bfseries \Large}{\thesection}{}{}[\vspace{-5mm}\rule{\titlefwidth}{1pt}]
\titleformat{\subsection}[runin]{\normalsize}{\thesubsection}{}{}
\titleformat{\subsubsection}{\normalsize}{\thesubsubsection}{}{}

% Spacing
\titlespacing{\section}{\titleleft}{5mm}{1mm}
\titlespacing{\subsection}{\titleleft}{3mm}{1mm}
\titlespacing{\subsubsection}{0cm}{2mm}{1mm}

% Custom headers
\newcommand{\workplace}[1]{\textsl{#1}}
\newcommand{\duration}[1]{\textbf{#1}}
\newcommand{\theplace}[2][]{\subsection*{\textbf{#2}#1}}
\newcommand{\thework}[2]{\subsubsection*{\workplace{#1}\hfill\duration{#2}}\par}
\newcommand{\thebold}[1]{\subsection*{\textbf{#1}}\vspace{-1mm}}

% Colors [http://ctan.uib.no/macros/latex/contrib/xcolor/xcolor.pdf]
\usepackage[svgnames]{xcolor}

% Hyperlinking
\usepackage[pdftex]{hyperref}
\hypersetup{
	colorlinks = true,
	linkcolor  = NavyBlue,
	urlcolor   = NavyBlue
}

% Footer
\usepackage{lastpage}
\usepackage{fancyhdr}
\pagestyle{fancy}
\renewcommand{\headrulewidth}{0pt}
\cfoot{\thepage\ / \pageref{LastPage}}
% }}}

\begin{document}

% Name
\hspace{\titleleft}\parbox{\titlefwidth}{
	{\bfseries \LARGE \name}\vspace{-2mm}\par
	\rule{\titlefwidth}{1pt}\vspace{2mm} % ruler
}

% Address information
\begin{tabular*}{\textwidth}{@{}l l @{\extracolsep{\fill}} r @{}}
{\bf Date of birth:} & 21 January 1989     & Forskarbacken 19, lgh 1102\\
{\bf Phone number:}  & +46 (0)70-283 55 65 & 114 15 Stockholm, Sweden \\
{\bf Email address:} & \href{mailto:victorha@kth.se}{\tt victorha@kth.se} &
\end{tabular*}

\section*{Education}
\theplace[, Stockholm, Sweden]{\href{http://kth.se}{Kungliga Tekniska Högskolan (The Royal Institute of Technology)}}

\thework{Master of Science degree in Computer Science}{2008.08--present}
Completed bachelor's degree and ongoing master's degree in computer science.

\theplace[, Täby, Sweden]{\href{http://www.taby.se/ava}{Åva Gymnasium}}

\thework{Technological education program}{2005.08--2008.06}
Three years of study with focus on math and computer science in gymnasium (upper secondary school).

\section*{Experience}
\theplace[, Stockholm, Sweden]{\href{http://www.polisforbundet.se}{Polisförbundet (The Swedish Police Union)}}

\thework{Web developer}{2011.03--present}
Developed and launched a WordPress-based web site for \href{http://www.polistidningen.se}{Polistidningen}, a magazine for members of the Swedish Police Union. The work consisted of implementing a web design by specification along with developing a number of custom functionalities specific to the site. I have also maintained and configured their web servers (in both Windows and Linux environments) since the launch.

\theplace[, Stockholm, Sweden]{\href{http://www.sverigesingenjorer.se}{Sveriges Ingenjörer (The Swedish Association of Graduate Engineers)}}

\thework{Web developer and designer}{2010.05--2010.12}
Designed and implemented a new design for \href{http://www.ingenjorsbloggen.se}{Ingenjörsbloggen}, the official blog of Sveriges Ingenjörer. The project, developed as a WordPress theme, also included implementation of social network functionalities.

\thework{Web developer and designer}{2009.09--present}
In 2009 I developed and launched a new WordPress-based web magazine for \href{http://www.ingenjoren.se}{Ingenjören}, a regularly published magazine for members of Sveriges Ingenjörer. I have since then provided support and developed several additions for the site.

\theplace[, Stockholm, Sweden]{Tekniska Högskolans Studentkår}

\thework{Web developer och designer}{2008--2009}
Designade, utvecklade och underhöll Osqledarens (studentkårens tidning) webbsida, baserad på WordPress.

\theplace[, Solna]{Siemens IT Solutions and Services}

\thework{Användar- och systemtestare}{2008}
Arbetade under sommaren som testare för ett SAP-system under utveckling.

\theplace[, Upplands Väsby]{Siemens Power Generation}

\thework{Underhållning av datorsystem}{2005--2007}
Under loven arbetade jag med installation och underhåll av servrar och arbetsstationer.

\section*{Technical Expertise}
Jag har bred erfarenhet och kunskaper inom programmering, med fokus på webbutveckling.

\thebold{Programmering:} Java, JavaScript/jQuery, Ruby, PHP, C(++), grundläggande C\#.net och python.

\thebold{Tekniker:} XHTML, CSS, MySQL, Apache, Microsoft IIS, Subversion, Git, Regular Expressions samt LaTeX.

\thebold{Mjukvara:} WordPress, Ruby on Rails, Eclipse, Visual Studio, Adobe Photoshop, Microsoft Office och vim.

\thebold{Operativsystem:} Microsoft Windows, Unix/Linux samt Mac OS X.

\section*{Languages}
\thebold{Svenska} - modersmål.
\thebold{Engelska} - akademisk nivå.

\section*{Volunteer Experience}
\theplace[, Kungliga Tekniska Högskolan]{Datasektionen}

\thework{Informationsansvarig}{2010}
Under ett år arbetade jag som informationsansvarig på sektionen, en post där jag ansvarade för informationsspridningen inom, från och till dess medlemmar.

\thework{Fadder i mottagningen}{2009, 2010, 2011}
Jag har sedan mitt andra år på KTH medverkat som fadder under mottagningen för de nyantagna på Datateknik. Som mottagningspersonal arbetar man heltid under tre veckor för att välkomna de cirka 200 nyantagna.

\section*{Miscellaneous}
Jag innehar B-körkort sedan år 2007.

\end{document}

